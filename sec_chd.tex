We develop a data-structure that supports fast queries for \emph{efficient paths}. 
Specifically, given a graph $G=(V,E)$ and a maximum budget $B$, we construct a data-structure such that, for any $s,t\in V$ and $b\leq B$, we return the length of the shortest $(s,t)$-path with cost at most $b$, denoted $\dist(s,t|b)$.
The actual path can easily be recovered as discussed in \cref{ssec:hldef}, hence we focus on querying $\dist(s,t|b)$ only.

In Section \ref{ssec:hldef} we discussed that, if a graph $G$ has HD $h$, we can simultaneously bound the preprocessing time, storage requirements and query time for constructing hub labels as functions of $h$.
This suggests that for the construction of provably efficient hub labels for the CSP problem, we need an analogous property for the set of \emph{efficient paths}.
\begin{definition}[Constrained Highway Dimension] The constrained highway dimension (CHD) of $(G,\ell,c)$, denoted $h_c$, is the HD of the efficient-path system $\PE$.
\end{definition}
Note that, since every shortest path is efficient, $h_c\geq h$.

We now have two main issues with this definition: first, it is unclear how this can be used to get hub labels, and second, it is unclear how the corresponding hub labels compare with those for shortest-path computations. 
To address this, we first convert efficient paths in $G$ to shortest paths in a larger \emph{augmented graph}. 
In Section \ref{ssec:hlcsp}, we use this to construct hub labels for CSP queries whose storage and query complexity can be bounded as $Bh_c$ (which can be strengthened further to $g(b)h_c$, where $g(b)$ measures the size of the Pareto frontier, cf. Section~\ref{sec:frontier}.). 
Finally, in Section \ref{ssec:hdvschd}, we show that the hub labels for CSP queries can in fact be related to the hub labels for SP queries under an additional natural condition on the efficient paths.