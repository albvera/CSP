The linear dependence on $B$ in the bound on HL sizes (cf. \cref{theo:HLeff}) is somewhat weak. 
Essentially, this corresponds to a worst-case setting where the efficient paths between any pair of nodes is different for each budget level. 
In most practical settings, changing the budget does not change the paths too much, and ideally the hub label sizes should reflect this fact. 
This is achieved via a more careful construction of hub labels, resulting in the following bound. 
\begin{theorem}\label{thm:markedhubs}
Let $(G,\ell,c)$ as in \cref{theo:HLeff} and $g:\N\to \N$ be such that, for every $s,t\in V$, $b\in \N$, $\abs{\crl{P\in\Pst^E: c(P)\leq b}} \leq g(b)$.
Then, we can construct hub labels of size $O(g(B)h_c\log D)$, and answer queries with budget $b$ in time $O(g(b)h_c\log D)$.
\end{theorem}

Note that there always exists such a function $g$ and the worst case is $g(b)=b$.
The proof depends on a different technique for constructing HL (refer to \cref{alg:forwardhub,alg:reversehub} in \cref{sec:proofs}). 
The main idea is to sort the efficient paths for each source node $s$ by cost, and then carefully mark nodes when they are added to the forward HL; these marked nodes are then used to construct the reverse HL.
For brevity, the complete algorithmic details and proof are deferred to \cref{sec:proofs}.