We consider a directed graph $G=(V,E)$, where each edge $e\in E$ has an associated \emph{length} $\ell(e)\in\N_+$, and \emph{cost} $c(e)\in\N_+\cup\crl{0}$.
The triplet $(G,\ell,c)$ is called a \emph{network}.
For any path $P$, we define its length $\ell(P)$ and cost $c(P)$ as the sum of edge lengths and edge costs in $P$. 
Our goal is to develop a data structure to answer \emph{Constrained Shortest-Path} (CSP) queries efficiently as described next.

\begin{problemdef}[framed]{Constrained Shortest-Path Queries}
	Input: & Graph $G=(V,E)$, costs $c$, lengths $\ell$, and maximum budget $B$.\\
	Preprocessing Task:& Create a data structure $\mathcal{S}$. \\
	Real Time Task:& For any source-terminal pair $s,t\in V$ and budget $b\leq B$, use $\mathcal{S}$ to return a path solving $\min\crl{\ell(P):c(P) \leq b, P \text{ is an } (s,t)\text{-path}}$.
	The triplet $(s,t,b)$ defines a \emph{query} and it is arbitrarily specified by the user. \\
  	Performance Metrics: &  Size of $\mathcal{S}$, preprocessing time (to compute $\mathcal{S}$), and query time (to return a path given $s,t,b$). 
\end{problemdef}


Note that the two computation times are in different scales.
Indeed, it is acceptable to have a preprocessing time of a few minutes, but the query time should be in milliseconds.
Observe also that the performance metrics are in conflict.
To illustrate this, define $n=\abs{V}$ as the number of nodes and consider two extreme cases.
(i) for each $(s,t,b)$ we store in $\mathcal{S}$ the solution to the associated query, which has $O(1)$ query time, but space $\abs{\mathcal{S}}=\Omega(Bn^2)$.
(ii) use no data structure, which consumes no space, but requires $\Omega(bn\log n)$ query time using Dijkstra (see \cref{prop:shorteffic}).

We stress that either storage $\Omega(n^2)$ or query time $\Omega(n)$ is unacceptable for modern applications.
The goal is to find $\mathcal{S}$ in polynomial time with small storage and fast query time.

\medskip
\paragraph{Our Results.}
We identify a parameter $\hc$, called the Constrained Highway Dimension (CHD), which generalizes the concept introduced in \citep{hd_journal}, and obtain the results below.
In each case, the data structure can be computed in polynomial time.
$D$ represents the diameter of the graph.
\begin{enumerate}
\item We give a data structure of size $\tilde O(nB\cdot B\hc\log D)$ and query time $\tilde O(b\hc\log D)$ for any $(s,t,b)$, see \cref{theo:HLeff,thm:markedhubs}.
We relate $\hc$ to the HD in \cref{theo:witness_doubling}.
\item We introduce the notion of \emph{Average CHD}, which is strictly weaker than CHD, and show that we can obtain a data structure with the same guarantees, but the query time is in average over $(s,t)$, see \cref{theo:preproc_avg}.
\item Building on the average CHD, we give a condition, interpreted as having few overpasses, such that the query time is  $\tilde O(bh\alpha\log D)$, average over $(s,t)$, with space requirement $\tilde O(n B\cdot Bh\alpha\log D)$, where $h$ is the HD and $\alpha$ the doubling constant, see \cref{theo:overpasses}.
We observe that, for road networks, it is conjectured that $h=\text{polylog}(n)$ and $\alpha=O(1)$, hence our result explains the good empirical performance of our algorithm, see \cref{sec:numeric}.
\end{enumerate}


\medskip
\paragraph{Final Setup Details.}
For any source-terminal pair $s,t\in V$, we denote by $\Pst$ the set of all simple $(s,t)$-paths (without loops or cycles). 
Throughout this work, we only consider simple paths, which we refer to as paths for brevity.
We denote the shortest $(s,t)$-path (if it exists) as $P(s,t)$, and denote the set of all shortest paths in $G$ as $\PS$.

For each node $v$, we denote its degree $\Delta(v)$ as the sum of the in-degree and out-degree, and define the \emph{maximum degree} $\Delta \defeq \max_v\Delta(v)$.
For $s,t\in V$, the distance from $s$ to $t$, denoted $\dist(s,t)$, is the smallest length among all paths $P\in\Pst$.
We define $\dist(s,t|b)$ to be the length of the shortest path with cost at most $b$, i.e., the length of the path returned by the query $(s,t,b)$.
If there is no feasible solution, we define $\dist(s,t|b)=\infty$.

For a node $v$ and a path $P$, we abuse notation to denote $\dist(v,P)$ as the minimum distance from $v$ to any node $w\in P$; the distance $\dist(P,v)$ from $P$ to $v$ is defined analogously.
Note that  $\dist(P,v)$ and  $\dist(v,P)$ need not be the same as the graph is directed.
We define $D\defeq\max_{P\in\PS}\ell(P)$ to be the diameter of $G$.


For $r>0$ and $v\in V$, we define the \emph{forward and reverse balls of radius $r$} by $\Bf_r(v)\defeq\{u\in V: \dist(v,u)\leq r\}$ and $\Bb_r(v)\defeq\{u\in V: \dist(u,v)\leq r\}$, and also define $B_r(v)\defeq\Bf_r(v)\cup\Bb_r(v)$.
Finally, a graph $G$ is said to have a \emph{doubling constant} $\alpha$ if, for any node $v$ and any $r>0$, the ball $B_{2r}(v)$ can be covered by at most $\alpha$ balls of radius $r$.