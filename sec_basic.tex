%!TEX root = main_vldb.tex

We consider a directed graph $G=(V,E)$ with \emph{length function} $\ell:E\to \N_+$, and \emph{cost function} $c:E\to \N_+\cup\{0\}$.
For each node $v$, we denote its degree $\Delta(v)$ as the sum of the in-degree and out-degree.
Finally, we define the \emph{maximum degree} $\Delta = \max_v\Delta(v)$.


A simple \emph{path} $P$ in $G$ is an acyclic sequence of nodes  $u_1u_2u_3\ldots u_k$ with $u_iu_{i+1}\in E$.
For any source-terminal pair $s,t\in V$, we denote by $\Pst$ the set of all simple paths from $s$ to $t$.
Throughout this work, we only consider simple paths, which for brevity, we henceforth refer to as paths.

The length $\ell(P)$ is the sum of edge lengths in $P$. Note also that any path $P$ with more than one node has length at least $1$ (since $\ell:E\to \N_+$).
For $s,t\in V$, the distance from $s$ to $t$, denoted $\dist(s,t)$, is the smallest length among all paths $P\in\Pst$.
The distance from a node $v$ to a path $P$, denoted $\dist(v,P)$, is measured as the minimum distance from $v$ to a node $w\in P$.
The distance to $v$, $\dist(P,v)$, is defined analogously.\footnote{Note that  $\dist(P,v)$ and  $\dist(v,P)$ need not be the same as the graph is directed.}
We denote the shortest $(s,t)$-path (if it exists) as $P(s,t)$, and denote the set of all shortest paths in $G$ as $\PS$.
Finally, we define $D=\max_{P\in\PS}\ell(P)$ to be the diameter of $G$.


Analogous to length, we define the cost $c(P)$ as the sum of edge costs on path $P$. 
Our goal is to develop a data-structure that can solve the following \emph{Constrained Shortest-Path} (CSP) problem: Given a source-terminal pair $s,t$ and a budget $b$, we want to return a path $P$ that solves
\begin{align*}
\min_{P \in \Pst}  \quad \ell(P) \quad
\text{s.t. } \quad  c(P) \leq b 
\end{align*}
We define $\dist(s,t|b)$ to be the minimum of this problem.
If there is no feasible solution, we define $\dist(s,t|b)=\infty$.
Note that the CSP problem may have multiple solutions (with different feasible costs), as there could be several paths with the same length and cost lower than $b$.
To limit these solutions to those with minimal cost, we require that the path also be \emph{efficient}. 
\begin{definition}
\textbf{(Efficient Path)} A path $P\in \Pst$ is called \emph{efficient} if there is no other path $P'\in \Pst$ such that $\ell(P')\leq \ell(P)$ and $c(P')\leq c(P)$ with at least one inequality strict.
\end{definition}
We denote the set of all efficient paths as $\PE$, and define the \emph{Pareto frontier} from $s$ to $t$ as $\Pst\cap\PE$.
Observe that every subpath of an efficient path is also efficient (if not, we could improve the path by replacing the subpath).

For $r>0$ and $v\in V$, we define the \emph{forward and reverse balls of radius $r$} by $\Bf_r(v)\defeq\{u\in V: \dist(v,u)\leq r\}$ and $\Bb_r(v)\defeq\{u\in V: \dist(u,v)\leq r\}$, and also define $B_r(v)\defeq\Bf_r(v)\cup\Bb_r(v)$.
Graph $G$ is said to have a \emph{doubling dimension} $\alpha$ if, for any node $v$ and any $r>0$, the ball $B_{2r}(v)$ can be covered by at most $\alpha$ balls $B_r(w)$ of radius $r$. 