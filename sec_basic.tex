We consider a directed graph $G=(V,E)$, where each edge $e\in E$ has an associated \emph{length} $\ell(e)\in\N_+$, and \emph{cost} $c(e)\in\N_+\cup\crl{0}$.
For each node $v$, we denote its degree $\Delta(v)$ as the sum of the in-degree and out-degree, and define the \emph{maximum degree} $\Delta \defeq \max_v\Delta(v)$.
For any source-terminal pair $s,t\in V$, we denote by $\Pst$ the set of all simple $(s,t)$-paths (without loops or cycles). 
Throughout this work, we only consider simple paths, which we refer to as paths for brevity.

For any path $P$, we define its length $\ell(P)$ and cost $c(P)$ as the sum of edge lengths and edge costs in $P$. 
Note that any path $P$ with more than one node has length at least $1$ (since we assume lengths are integers).
For $s,t\in V$, the distance from $s$ to $t$, denoted $\dist(s,t)$, is the smallest length among all paths $P\in\Pst$.
For a node $v$ and a path $P$, we abuse notation to denote $\dist(v,P)$ as the minimum distance from $v$ to any node $w\in P$; the distance $\dist(P,v)$ from $P$ to $v$ is defined analogously.
Note that  $\dist(P,v)$ and  $\dist(v,P)$ need not be the same as the graph is directed.
We denote the shortest $(s,t)$-path (if it exists) as $P(s,t)$, and denote the set of all shortest paths in $G$ as $\PS$.
Finally, we define $D\defeq\max_{P\in\PS}\ell(P)$ to be the diameter of $G$.

Our goal is to develop a data-structure to answer \emph{Constrained Shortest-Path} (CSP) queries efficiently: 
Given a source-terminal pair $s,t$ and a budget $b$, we want to return a path $P$ solving
\[
\min\crl{\ell(P):c(P) \leq b, P \in \Pst }.
\]
We define $\dist(s,t|b)$ to be the minimum of this problem.
If there is no feasible solution, we define $\dist(s,t|b)=\infty$.
Note that the CSP problem may have multiple solutions as there could be several paths with the same length and cost lower than $b$.
To limit these solutions to those with minimal cost, we require that the path also be \emph{efficient}. 
\begin{definition}[Efficient Path]
\label[definition]{def:effpath}
A path $P\in \Pst$ is called \emph{efficient} if there is no other path $P'\in \Pst$ such that $\ell(P')\leq \ell(P)$ and $c(P')\leq c(P)$ with at least one inequality strict.
\end{definition}
We denote the set of all efficient paths as $\PE$.
Efficient $(s,t)$-paths form what is known as the \emph{Pareto frontier}, $\Pst\cap\PE$.
Observe that every subpath of an efficient path is also efficient (if not, we could improve the path by replacing the subpath).
For $r>0$ and $v\in V$, we define the \emph{forward and reverse balls of radius $r$} by $\Bf_r(v)\defeq\{u\in V: \dist(v,u)\leq r\}$ and $\Bb_r(v)\defeq\{u\in V: \dist(u,v)\leq r\}$, and also define $B_r(v)\defeq\Bf_r(v)\cup\Bb_r(v)$.
Finally, a graph $G$ is said to have a \emph{doubling dimension} $\alpha$ if, for any node $v$ and any $r>0$, the ball $B_{2r}(v)$ can be covered by at most $\alpha$ balls of radius $r$.