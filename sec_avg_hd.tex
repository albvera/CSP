When building systems to answer queries, the essential metric is the average performance; worst case analysis is too pessimistic and not relevant.
To capture the average performance, we relax the definition of HD in two ways.
First, a LSHS needs to be locally sparse just ``in average''.
Second, we forget the strong limitation of hitting $S_r(v,\calQ)$ and ask only for the existence of a multiscale LSHS.

\begin{definition}[Average LSHS]
Given $r>0$ and a system $\calQ$, a set $C\subseteq V$ is an average $(h,r)$-LSHS if it hits $\calQ_r$ and is locally sparse in average, i.e.,
$\frac{1}{n}\sum_{v\in V} \abs{B_{2r}(v)\cap C} \leq h$.
\end{definition}

\begin{definition}[Average HD]
The system $\calQ$ has average HD $h$ if, for every $r>0$, there exists an average $(h,r)$-LSHS.
\end{definition}

From the definition is clear that Average HD is a strictly weaker property than HD.
Nevertheless, this is enough to parametrize the average behaviour of HL.
Notice that, since query time depends linearly on the hub size, the next result implies both storage and performance bounds.

\begin{theorem}\label{theo:preproc_avg}
If $\PS$ has average HD $h$, then we can obtain, in polynomial time, HL such that 
\[
\frac{1}{n}\sum_{v\in V} \abs{\Lf(v)} \leq h'\log D \quad \text{ and }\quad
\frac{1}{n}\sum_{v\in V} \abs{\Lb(v)} \leq h'\log D,
\]
where $h'=\Or(\Delta h\log (hn\Delta))$.
\end{theorem}
\begin{proof}
We only go over the preprocessing, since the construction does not change and the bound for the size easily follows. 
The objective is to obtain a set $C_i$ which is an average $(h',2^i)$-LSHS.
This is a minimum cost hitting set problem.
Indeed, we want to solve
\[
\min_{C\subseteq V} \sum_{v\in V}\abs{B_{2r}(v)\cap C}  \quad \text{ s.t. } \quad C \text{ hits } \PS_r.
\]
To see the equivalence, we assign to each node $u$ the cost $c(u)=\abs{\crl{v\in V: u\in B_{2r}(v)}}$.
Given a minimum cost hitting set problem with optimum value $\tau$, if the set system has VC-dimension $d$, the algorithm in \cite{vc_dim_hitting} finds a solution, in polynomial time, with cost at most $\Or(d\tau\log(d\tau))$.

By assumption, the minimum of the problem is at most $hn$.
Now we do the same mapping as before, where the ground set is changed to $E$, hence the VC-dimension is 2 and now the minimum is at most $h\Delta n$.
We apply the algorithm in \cite{vc_dim_hitting} and obtain a solution $C_i$ with cost at most $\Or(h\Delta n\log(h\Delta n))$.
We have obtained an average $(h',r)$-LSHS as desired.
\end{proof}

\begin{remark}
The algorithm in \cref{theo:preproc_avg} makes one call to the VC-dimension solver for each $C_i$.
On the other hand, the algorithm in \cite{highway2013} calls up to $n$ times the solver for each $C_i$.
Finally, there is an extra $\log n$ factor in the approximation guarantee, but now the value of $h$ can be much smaller.
\end{remark}