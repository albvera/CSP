Converting the partial-witness condition to a more interpretable condition is difficult in general, as the structure of $\PS$ and $\PE$ may be complex. 
One way to get such a condition, however, is by considering \emph{average-case} performance metrics.
For this, we relax the definition of HD in two ways: $(i)$ we require LSHS to be locally sparse ``on average'' over all nodes, and
$(ii)$ we only require the existence of LSHS (as opposed to a hitting set for $S_r(v,\calQ)$).

\begin{definition}[Average LSHS]
Given $r>0$ and a system $\calQ$, a set $C\subseteq V$ is an average $(h,r)$-LSHS if it hits $\calQ_r$ and is locally sparse in average, i.e.,
$\frac{1}{n}\sum_{v\in V} \abs{B_{2r}(v)\cap C} \leq h$.
\end{definition}
\begin{definition}[Average HD]
The system $\calQ$ has average HD $h$ if, for every $r>0$, there exists an average $(h,r)$-LSHS.
\end{definition}

From the definition is clear that average HD is a strictly weaker property than HD; in particular, as we discuss before, the existence of LSHS does not guarantee their efficient computation.
Nevertheless, the above definition turns out to suffice to parametrize the average behavior of HL, as we show next.
Note that, since query time depends linearly on the hub size, the next result implies both storage and performance bounds.
\begin{theorem}\label{theo:preproc_avg}
If $\PS$ has average HD $h$, then we can obtain, in polynomial time, HL with average size 
$\frac{1}{n}\sum_{v\in V} \abs{\Lf(v)} \leq h'\log D$ and 
$\frac{1}{n}\sum_{v\in V} \abs{\Lb(v)} \leq h'\log D$,
where $h'=\Or(\Delta h\log (hn\Delta))$.
\end{theorem}
\begin{proof}
We only show how to compute average LSHS, since the construction of HL is the same as in \cref{theo:construct_hl} and the bound for the size easily follows. 
The objective is to obtain a set $C_i$ which is an average $(h',2^i)$-LSHS.
This turns out to be a minimum-cost hitting set problem.
Indeed, we want to solve
\[
C_i = \arg\!\min\crl*{\sum_{v\in V}\abs{B_{2^{i+1}}(v)\cap C}: C\subseteq V, C \text{ hits } \PS_{2^i}}.
\]
This follows from a symmetry argument, assigning to each node $u$ the cost $c(u)=\abs{\crl{v\in V: u\in B_{2^{i+1}}(v)}}$.
On the other hand, given a minimum cost hitting set problem with optimum value $\tau$, if the set system has VC-dimension $d$, the algorithm in \citep{vc_dim_hitting} finds a solution, in polynomial time, with cost at most $\Or(d\tau\log(d\tau))$.

By assumption, the minimum of the problem is at most $hn$.
Now we perform a mapping from paths to sets, where the ground set is $E$ and paths are sequences of edges.
This system has VC-dimension 2, and now the minimum is at most $h\Delta n$.
We apply the algorithm in \citep{vc_dim_hitting} to obtain a solution $C_i$ with cost at most $nh'$; this is the promised average $(h',2^i)$-LSHS.
\end{proof}
