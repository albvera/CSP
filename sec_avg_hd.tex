The partial-witness property and \cref{theo:witness_doubling} provide a nice link between the CHD and HD, essentially showing that scaling CSP queries is of similar complexity to scaling SP queries as long as any efficient path contains large shortest-path segments. 
As an example, consider a \emph{multimodal mapping service} which gives transit routes combining walking, bus and subway, while ensuring at most $k$ transfers. 
For $k=3$, each route has at most $4$ segments, which gives use the partial witness property with $\beta = 2$. 
Now if each individual network has small HD, then the CHD is also small.

Converting the partial-witness condition to a more interpretable condition is difficult in general, as the structure of $\PS$ and $\PE$ may be complex. 
One way to get such a condition, however, is by considering \emph{average-case} performance metrics.
For this, we relax the definition of HD in two ways: $(i)$ we require LSHS to be locally sparse ``on average'' over all nodes, and
$(ii)$ we only require the existence of LSHS (as opposed to a hitting set for $S_r(v,\calQ)$).

\begin{definition}[Average LSHS]
Given $r>0$ and a system $\calQ$, a set $C\subseteq V$ is an average $(h,r)$-LSHS if it hits $\calQ_r$ and is locally sparse in average, i.e.,
$\frac{1}{n}\sum_{v\in V} \abs{B_{2r}(v)\cap C} \leq h$.
\end{definition}
\begin{definition}[Average HD]
The system $\calQ$ has average HD $h$ if, for every $r>0$, there exists an average $(h,r)$-LSHS.
\end{definition}

From the definition is clear that average HD is a strictly weaker property than HD; in particular, as we discuss before, the existence of LSHS does not guarantee their efficient computation.
Nevertheless, the above definition turns out to suffice to parametrize the average behavior of HL:
\begin{theorem}\label{theo:preproc_avg}
If $\PS$ has average HD $h$, then we can obtain, in polynomial time, HL with average size 
$\frac{1}{n}\sum_{v\in V} \abs{\Lf(v)} \leq h'\log D$ and 
$\frac{1}{n}\sum_{v\in V} \abs{\Lb(v)} \leq h'\log D$,
where $h'=\Or(\Delta h\log (hn\Delta))$.
\end{theorem}
Note that since query time depends linearly on the hub size, the above result implies both storage and performance bounds.
%\begin{proof}
%We only go over the preprocessing, since the construction is the same as in \cref{theo:construct_hl} and the bound for the size easily follows. 
%The objective is to obtain a set $C_i$ which is an average $(h',2^i)$-LSHS.
%This turns out to be a minimum cost hitting set problem.
%Indeed, we want to solve
%$
%\min\crl{\sum_{v\in V}\abs{B_{2r}(v)\cap C}: C\subseteq V, C \text{ hits } \PS_r}.
%$
%This follows from a symmetry argument, assigning to each node $u$ the cost $c(u)=\abs{\crl{v\in V: u\in B_{2r}(v)}}$.
%On the other hand, given a minimum cost hitting set problem with optimum value $\tau$, if the set system has VC-dimension $d$, the algorithm in \cite{vc_dim_hitting} finds a solution, in polynomial time, with cost at most $\Or(d\tau\log(d\tau))$.
%
%By assumption, the minimum of the problem is at most $hn$.
%Now we perform a mapping, where the ground set is changed to $E$ and paths are sequences of edges instead of nodes.
%This system still has VC-dimension 2, and now the minimum is at most $h\Delta n$.
%We apply the algorithm in \cite{vc_dim_hitting} and obtain a solution $C_i$ with cost at most $\Or(h\Delta n\log(h\Delta n))$; this gives the promised average $(h',r)$-LSHS.
%\end{proof}
