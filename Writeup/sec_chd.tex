We now turn to the problem of constructing hub-labels for finding constrained shortest-paths (and more particularly, for computing efficient paths) in $G$. 
Recall we are interested in settings where each edge in the network is endowed with a length and a cost, where we assume all edge-costs are non-negative integers.
Now, we want to develop a data-structure that supports fast queries for \emph{efficient paths}: given any pair $(s,t)$ and an (integer) budget $B$, we want to return a path $P\in \calP_{s,t}$ such that there is no other path $P'\in \calP_{s,t}$ with $\ell(P')\leq \ell(P)$ and $c(P')\leq c(P)$ (with at least one inequality strict). 


As in Section \ref{ssec:hldef}, we will use hub labels to enable fast querying of  we discuss how if a graph $G$ has HD $\leq h$, then we can simultaneously bound the preprocessing time, storage requirements and query time of a certain algorithm for constructing hub-label as functions of $h$.
This suggests that for the construction of provably efficient HL for the CSP problem, one needs to define an analogous property for the set of \emph{efficient paths}.
To this end, we have
\begin{definition}
\textbf{(Constrained Highway Dimension)}\\ The constrained highway dimension (CHD) of $(G,\ell,c)$, denoted $h_c$, is the HD of the efficient-path system $\PE$.
\end{definition}
Note that every shortest path is efficient, and thus $h_c\geq h$ (where $h$ is the HD of $G$ under shortest paths).


We now have two main issues with this definition: First, it is unclear how this can be used to get hub labels, and second, it is unclear how the corresponding hub labels compare with those for shortest-path computations. 




