\label{sec:ch}
We present here how to extend the concept of HD in order to prove the efficiency of CH in directed graphs.

\begin{definition}
	For $r>0$, the shortest path $P\in\calP$ is $r$-significant if we can add at most one vertex at each end of $P$ to obtain a shortest path $P'\in\calP$ such that $\ell(P')>r$.
	The path $P'$ is called an $r$-witness of $P$.
	We denote $\calP_r$ as the set of all $r$-significant paths.
\end{definition}


Intuitively, a path is significant if it represents a long path.
Observe that, if $P\in \calP$ is such that $\ell(P)>r$, then $P$ is $r$-significant by definition.
We remark also that a path can have many $r$-witnesses.

\begin{definition}
	For $r>0$, an $(h,r)$-SPHS is a set $C\subseteq V$ satisfying: 
	\begin{enumerate}
		\item The set $\calP_r$ is hit by $C$: for every $P\in\calP_r$, $P\cap C\not=\varnothing$.
		\item $C$ is $h$-sparse: for every $v\in V$, $\card{\Bb_{2r}(v)\cap C}\leq h$ and $\card{\Bf_{2r}(v)\cap C}\leq h$.
	\end{enumerate}
\end{definition}

The following is not the usual notion of a sphere.
For $v\in V$, we are interested in paths, not points, close to $v$.
We do not say that $P$ is close to $v$ if $\dist(v,P)$ is small, but rather if $P$ has some $r$-witness $P'$ with $\dist(v,P')$ small.


\begin{definition}
	For $v\in V, r>0$, the path neighbourhood  $\Sf_r(v)$ is defined as the set of paths $P\in\calP_r$ such that $P$ has some $r$-witness $P'$ with $\dist(v,P')\leq 2r$.
	The neighbourhood $\Sb_r(v)$ is defined analogously.
\end{definition}

\begin{definition}
	The network $(G,\ell)$ has HD $h$ if, for every $r>0, v\in V$, there exists $H_{v,r}\subseteq V$ such that $\card{H_{v,r}}\leq h$ and $H_{v,r}$ hits every path in $S_r^+(v)$ and every path in $S_r^-(v)$.
\end{definition}


\subsection{Additional Data Requirements}

Given a rank in the nodes, the shortcut process works as in the non-directed case:
\begin{enumerate}
	\item Let $G'$ be a temporary copy of $G$.
	\item Remove nodes of $G'$ and its edges in increasing rank.
	\item When removing $v$, if some unique shortest path in $G$ uses $uvw$, add $(u,w)$ to $G'$ with length $\ell(u,v)+\ell(v,w)$.
\end{enumerate}

Call $E^+$ the set of edges created in the shortcut process.
A source-destination query runs bidirectional Dijkstra, but each search only considers paths of increasing ranks.

As in the non-directed case, let $Q_i=C_i\setminus \cup_{j>i}C_j$ be the partition of $V$.
All the ranks in $Q_i$ are smaller than those in $Q_{i+1}$, within each $Q_i$ the rank is arbitrary.

\begin{lemma}\label{lemma:intshort}
	Let $P$ be a shortest path in the original graph.
	If $P$ has at least three vertices and $\ell(P)>2^{\gamma}$, then some internal vertex of $P$ belongs to a level $Q_x$, $x>\gamma$.
\end{lemma}
\begin{proof}
	The path $P'$ obtained by removing the endpoints of $P$ is $\ell(P)$-significant.
	By definition of the $C_i$'s, $C_{\gamma+1}$ hits $P'$ at some node $u$.
	By construction of the partition, $u\in Q_x$ with some $x>\gamma$. 
\end{proof}

Now we show that each node adds at most $h$ to its out-degree for each $Q_i$, so the process adds at most $h\log D$ to the out-degree of each node.
\begin{lemma}
	Assume the network admits the $C_i$'s.
	For any $v$ and fixed $j$, the number of shortcuts $(v,w)$ with $w\in Q_j$ is at most $h$.
\end{lemma}
\begin{proof}
	Let $i$ be the level such that $v\in Q_i$ and define $\gamma:=\min(i,j)$.
	We claim that $w\in \Bf_{2^\gamma}(v)$.
	Assume the claim, then the number of shortcuts is at most $\card{Q_j\cap \Bf_{2^\gamma}(v)}$, but using local sparsity and set inclusion:
	\[
	\card{Q_j\cap \Bf_{2^\gamma}(v)}\leq \card{C_j\cap\Bf_{2\cdot 2^{j-1}}(v)} \leq h.
	\]
	
	All that remains is to prove the claim.
	The shortcut $(v,w)$ was created when the process removed the last internal vertex of the shortest path $P(v,w)$ in $G$.
	Necessarily all the internal vertices are in levels at most $\gamma$, because they were removed before $v$ and $w$, hence they have lower rank.
	Finally, apply Lemma~\ref{lemma:intshort} to conclude that $\ell(P(v,w))\leq 2^\gamma$.
\end{proof}

We need to bound the in-degree, because it could be that some node $v$ is receiving many edges.
The proof is basically the same.
\begin{lemma}
	Assume the network admits the $C_i$'s.
	For any $v$ and fixed $j$, the number of shortcuts $(w,v)$ with $w\in Q_j$ is at most $h$.
\end{lemma}
\begin{proof}
	Same as in the previous lemma, but now $w\in \Bb_{2^\gamma}(v)$.
\end{proof}

We can conclude now that the number of shortcuts, i.e.\ $\card{E^+}$, is at most $2nh\log D$.

\subsection{Query Time}

As we mentioned before, the query performs Dijkstra from the source and target, but always constructing paths of increasing rank.
When scanning a vertex $v$, the forward search has a label $\dist(s,v)'$. 
The labels always satisfy $\dist(s,v)'\geq\dist(s,v)$, but, since the algorithm only goes to higher ranks, equality is not guaranteed.

We add a pruning rule analogous to the non-directed case: when the forward search scans a node $v$, if $(v,w)\in E\cup E^+$ and $w\in Q_i$, then $w$ is added to the priority queue only if $\text{rank}(w)>\text{rank}(v)$ and $\dist(s,v)'+\ell(v,w)\leq 2^i$.
For the backward search, the condition is the analogous $\dist(v,t)'+\ell(w,v)\leq 2^i$ when $(w,v)\in  E\cup E^+$.

\begin{proposition}
	The query with additional pruning returns the correct distance.
	Additionally, each Dijkstra scans at most $h$ nodes in each level.
\end{proposition}
\begin{proof}
	Let us analyse the forward search.
	Say the node $v$ is being scanned, $w\in Q_i$ is a candidate and $\dist(s,v)'+\ell(v,w)>2^i$.
	If the current path $P'$ to $w$ is optimal, then $P(s,w)$ is $2^i$-significant and it is hit by $C_{i+1}$. 
	As a consequence, $P(s,w)$ contains an internal vertex with higher rank than $w$.
	This vertex cannot be in $P'$ nor a shortcut containing it, thus contradicting the optimality of $P'$.
	We conclude that $P'$ is not optimal and $w$ can be ignored.
	
	Bounding the number of scanned nodes is easy; every $w\in Q_i$ added to the queue satisfies $w\in \Bf_{2^i}(s)$, so applying local sparsity we finish the proof.
\end{proof}

As a result, the forward search adds at most $h\log D$ nodes to the queue;
each of node amounts to $O(\text{outdeg} (G^+))$ operations, i.e., $O(\text{outdeg}(G) + h\log D)$ operations.
