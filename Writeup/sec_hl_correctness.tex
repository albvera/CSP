Arguably the two most successful algorithms for the SP problem are CH and HL.
When a network has small HD, it is possible to obtain efficient structures for both of these algorithms.
We explain here the construction for HL.
For the construction of CH see Section~\ref{sec:ch}.

The HL algorithm is described as follows.
Every node $v$ is assigned two sets of nodes, $\Lf(v)$ and $\Lb(v)$, referred to as forward and backward hubs.
We store $\dist(v,w),\dist(u,v)$ for every $w\in\Lf(v),u\in\Lb(v)$.
The hubs are said to satisfy the cover property if, for any $s\neq t\in V$, $\Lf(s)\cap\Lb(t)$ contains at least one node in $P(s,t)$.
In the case that $t$ is not reachable from $s$, it must be that $\Lf(s)\cap\Lb(t)=\varnothing$.

With the cover property it is trivial to obtain $\dist(s,t)$; we just compare all nodes $w\in\Lf(s)\cap\Lb(t)$ and return the minimum of $\dist(s,w)+\dist(w,t)$.
If the hubs are ordered by ID, we can obtain the distance with a single coordinated sweep in time $\Or(\card{\Lf(s)}+\card{\Lb(t)})$.
Note that the size of the HL bounds both the query time and the storage requirements.
The following result shows how to obtain small HL, the construction of hubs mimics the undirected case.

\begin{theorem}\label{theo:construct_hl}
If $(G,\ell)$ has HD $h$, then we can construct hub labels of size at most $h\log D$.
\end{theorem}
\begin{proof}
The hubs are defined as
\begin{align*}
\Lf(v):&=  \bigcup_{i=1}^{\log D}C_i\cap \Bf_{2^i}(v)\\
\Lb(v):&= \bigcup_{i=1}^{\log D}C_i\cap \Bb_{2^i}(v).
\end{align*}
Since $C_i$ is an $(h,2^{i-1})$-SPHS and we intersect with balls of radii $2\cdot 2^{i-1}$, every set in the union contributes at most $h$ elements and the maximum size is as claimed.

To prove the cover property, we note that, if $t$ is not reachable from $s$, by definition $\Lf(s)\cap\Lb(t)=\varnothing$.
This is because the elements in $\Lf(s)$ are reachable from $s$ and the elements in $\Lb(t)$ reach $t$.

In the case that $P(s,t)$ exists, take $i$ such that $2^{i-1}<\ell(P(s,t))\leq 2^i$.
The path $P(s,t)$ is hit by $C_i$ and any point in the path belongs to both $\Bf_{2^i}(s)$ and $\Bb_{2^i}(t)$.
Such point hitting $P(s,t)$ is then in both hubs, which shows the result.
\end{proof}

\subsubsection{Distance Oracles and Path Computation} \label{sec:path_oracl}
The previous construction is for distance oracles, i.e., they return the length of the SP, but not the path itself.
It is possible to recover the path with additional data; if we double the storage and compute not only the distance to $w\in\Lf(v)$, but also the second node in $P(v,w)$, then it is easy to recover the path.
Each time we run a HL query, at least one node $w\in P(s,t)$, $w\neq s,t$, is returned and we sequentially run queries for the subpaths $P(s,w)$ and $P(w,t)$.
Note that we do need to store this extra information, for example, it could be that $\Lf(s)\cap\Lb(t)=\{s\}$, so the process cannot continue without an additional node.