\anote{maybe not a subsection of CSP? it applies to any path system (e.g. shortest paths)}
It was shown in \cite{highway2013} that the set system of shortest paths has VC-dimension of $2$.
More formally, paths $v_1v_2\ldots v_k$ are mapped to sets $\{v_1,\ldots,v_k\}$ and this new set system has small VC-dimension.
Observe that, if $G$ is undirected, then each subset represents exactly two paths, namely the $(s,t)$-path and the $(t,s)$-path.
Therefore, a hitting set for the set system corresponds to a hitting set for $\calP$.

Recall that we assume uniqueness of shortest paths.
It is easy to see that the VC-dimension of the previous set system is 2.
Indeed, a path $abc$ is mapped to $\{a,b,c\}$ and the element $\{a,c\}$ would never form a part of the system; the shortest path between $a$ and $c$ must pass through $b$, therefore $\{a,b,c\}$ cannot be shattered.

In directed graphs the same relation can still be made, but the VC-dimension would not be small.
For example, the directed graph can contain the shortest paths $abc$ and $ca$, mapping to sets $\{a,b,c\}$ and $\{c,a\}$, breaking the previous argument.

To overcome this, we now think of paths as sequences of edges.
The ground set for the set system is $E$, instead of $V$ and a path $e_1e_2\ldots e_k$ is mapped to $\{e_1,e_2,\ldots,e_k\}$.
Note that each $\{e_1,e_2,\ldots,e_k\}$ corresponds uniquely to one path.
Recall that in a system $\calQ$ the cycles are not allowed.
We denote $\pi(Q)$ the set of edges in a path $Q$.

\begin{proposition}
Given a system $\calQ$, the corresponding set system $(E,\{\pi(Q):Q\in\calQ\})$ has VC-dimension 2.
\end{proposition}

We now show how to obtain polynomial time preprocessing.
Specifically, the algorithms we use are based on the existence of $(h,2^i)$-LSHS.
Assuming a system has HD $h$, we can obtain an approximation of these set, where the sparsity increases by the maximum degree $\Delta$. 

\begin{proposition}
If a system $\calQ$ has HD $h$, for some $h'=\Or(h\Delta\log(h\Delta))$ there exists a polynomial time algorithm to obtain a $(h',r)$-LSHS.	
\end{proposition}
\begin{proof}
Denote $S_r(v):=\Sf_r(v,\calQ)\cup\Sb_r(v,\calQ)$, i.e., $S_r(v)$ contains all the $r$-significant paths ``close'' to $v$. 
Observe that, for fixed $v\in V$, the set system $(E,\{\pi(Q):Q \in S_r(v)\})$ admits a hitting set of size $h\Delta$.
Indeed, we know that exists $H_{v,r}\subseteq V$, $\card{H_{v,r}}\leq h$, hitting every path in $\Sf_r(v,\calQ)$ and in $\Sb_r(v,\calQ)$.
The desired hitting set consists of all the edges adjacent to a node in $H_{v,r}$.

If the minimum size of a set system is $s$ and the VC-dimension is $d$, then the algorithm in \cite{vc_dim_hitting} obtains, in polynomial time, a hitting set of size at most $\Or(sd\log(sd))$.
In particular, we can use the algorithm to obtain a set $\tilde F_{v,r}\subseteq E$, of size at most $h'=\Or(h\Delta\log(h\Delta))$, hitting the set system $(E,\{\pi(Q):Q \in S_r(v)\})$ .

Consider the set $F_{v,r}\subseteq V$ that contains all the endpoints of edges in $\tilde F_{v,r}$.
It follows that $F_{v,r}\subseteq V$ can be obtained in polynomial time and is a hitting set for $S_r(v)$ of size $\card{F_{v,r}}\leq 2h'$.

Assume for now that we know the value of $h$.
Note that the value $h'$ can be computed from $h$ and the guarantee given by the oracle, i.e., the constant inside the big-O.
We construct the $(2h',r)$-LSHS iteratively.
At each iteration $i$ we maintain the following invariant: $C_i$ hits every path in $\calQ_r$.
In an iteration we check if $C_i$ is locally sparse, if not, we strictly reduce the cardinality of $C_i$ while maintaining the invariant.
Start with $C_0=V$. 
Let $B_{2r}(v):=\Bf_{2r}(v)\cup \Bb_{2r}(v)$.
Assume $v\in V$ is such that $\card{B_{2r}(v)\cap C_i}>2h'$ and let $C_{i+1}:=(C_i\setminus B_{2r}(v))\cup F_{v,r} $.
The cardinality strictly decreases and we only need to check the invariant.
Consider the paths hit by nodes removed in $C_i$, this set is
\begin{align*}
&\hspace{-1cm}\{Q\in\calQ_r: Q\cap C_i\cap B_{2r}(v)\neq \varnothing\}\\
&\subseteq \{Q\in\calQ_r: Q\cap B_{2r}(v)\neq \varnothing\} \subseteq S_r(v).
\end{align*}
Since $F_{v,r}$ hits $S_r(v)$, the proof is completed.

If we do not know the value of $h$, we can do a doubling search for $h'$. 
Indeed, if the guess of $h'$ is low, then at some point it could be that $\card{F_{v,r}}>2h'$, then we double $h'$ and restart the process.
\end{proof}
