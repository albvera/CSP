%!TEX root = main_vldb.tex

We now define the notions of hitting sets for path systems, and the highway dimension (HD), which we use to parametrize the performance of our algorithms. The highway dimension was introduced by Abraham et al.~\cite{highway2013,highway2010} for undirected graphs and the shortest-path set system. 
Our presentation closely follows these works.
However, we need to extend the definitions to directed graphs and general path systems for our purposes. 

We define a \emph{path system} $\calQ$ as any collection of paths.
Given a set $C\subseteq V$ and a path $Q$, we say that $C$ \emph{hits} $Q$ if some node in $Q$ belongs to $C$. Moreover, we say that $C$ is a \emph{hitting set for a path system} $\calQ$ if it hits every $Q\in\calQ$. Also, for any $r>0$, we say a path $Q$ is $r$-significant if $\ell(Q)>r$. 
For a given path-system $\calQ$, we denote $\calQ_r$ as the set of all $r$-significant paths in $\calQ$.

Hitting sets are useful for compressing path systems. In particular, even if the hitting set is large, the extent to which a path system can be compressed depends on the \emph{local sparsity} of hitting sets with respect to \emph{significant paths} of $\calQ$.

\begin{definition}
\label{def:lshs}
\textbf{(Locally-Sparse Hitting Sets)} Given a path system $\calQ$ and $r>0$, an $(h,r)$ locally-sparse hitting set (or $(h,r)$-LSHS) is a set $C\subseteq V$ with two properties: 
\begin{enumerate}[nosep]
\item Hitting: $C$ is a hitting set for $\calQ_r$.
\item Local sparsity: for every $v\in V$, $\abs*{B_{2r}(v)\cap C}\leq h$.
\end{enumerate}
\end{definition}


The existence of $(h,r)$-LSHS enables the compression of path-system $\calQ$ via the construction of \emph{hub labels} (cf. Section~\ref{ssec:hldef}). However, this existence does not guarantee the ability to efficiently compute it. 
To address this, Abraham et al.~\cite{highway2013} introduced the notion of the \emph{highway dimension} -- a property of the graph which ensures both existence and efficient computation of LSHS. 

To define the highway dimension, we first need two additional definitions:
for $v\in V, r>0$, we define the \emph{forward and reverse path-neighborhoods} with respect to path-system $\calQ$ as: 
\begin{align*}
S_r^+(v,\calQ) &\defeq\crl*{Q\in\calQ_r: \dist(v,Q)\leq 2r}, \\  
S_r^-(v,\calQ) &\defeq\crl*{Q\in\calQ_r: \dist(Q,v)\leq 2r}.
\end{align*}
As before, we have $S_r(v,\calQ) = S_r^+(v,\calQ)\cup S_r^-(v,\calQ)$. Now we can define the highway dimension (HD) of a path system $\calQ$. Essentially, the HD re-orders the sequence of qualifiers in the definition of $(h,r)$-LSHS: it requires the existence of a small hitting set for each individual neighborhood, rather than a single hitting set which is locally sparse. 
\begin{definition}
\label{def:hddef}
\textbf{(Highway Dimension)}
A path-system $\calQ$ has HD $h$ if, for every $r>0$, and for every $v\in V$, there exists a set $H_{v,r}\subseteq V$ such that $\card{H_{v,r}}\leq h$ and $H_{v,r}$ is a hitting set for $S_r^+(v,\calQ)\cup S_r^-(v,\calQ)$.
\end{definition}
The HD was defined in \cite{highway2013} for undirected graphs and only the shortest-path system $\PS$. 
The above definition is the natural analog for directed graphs and general path systems.
As shorthand, we refer to the HD of $(G,\ell)$ as that of $\PS$. 
Otherwise, it will be clear from the context the path system we are using.
To see that $HD\leq h$ is a more stringent requirement than the existence of an $(h,r)$-LSHS $C$, note that $C\cap B_{2r}(v)$ need not hit all the paths in $S_r(v,\calQ)$. 
However, if $G$ has $HD\leq h$, then this guarantees the existence of a $(h,r)$-LSHS.
\begin{proposition}
If the path system $\calQ$ has HD $h$, then, $\forall\,r>0$, there exists an $(h,r)$-LSHS.
\end{proposition}
This follows directly from Theorem 4.2 in \cite{highway2013}, which showed it for the shortest-path system. More importantly, as we discuss in Section~\ref{sec:preproc}, if $G$ has HD$\leq h$, then this permits efficient computation of locally-sparse hitting sets.
In particular, for any $r$, a simple greedy algorithm gives an $(\Or(h\log n),r)$-LSHS, and a more complex algorithm (which is still poly-time) gives an $(\Or(h\Delta\log h\Delta),r)$-LSHS.


Finally, we note that for pedagogical reasons, Definition~\ref{def:hddef} differs slightly from \cite{highway2013}.
We consider a less restrictive definition of path neighborhoods (albeit for general path systems) that is appropriate for our needs. Consequently the highway dimension as returned by our definition is smaller than that of \cite{highway2013}.\footnote{In particular, unlike \cite{highway2013}, the HD of $G$ as per our definition is not an upper bound to the maximum degree $\Delta$ or the doubling dimension $\alpha$.}
We discuss the relation between the two definitions in detail in Appendix \ref{app:generalhd}. 

