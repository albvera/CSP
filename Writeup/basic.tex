%!TEX root = main_vldb.tex

We consider a directed graph $G=(V,E)$ with \emph{length function} $\ell:E\to \N$, and \emph{cost function} $c:E\to \N\cup\{0\}$.
For each node $v$, we denote its in-neighborhood as $\Nb$ and out-neighborhood as $\Nf$; the node degree $\Delta(v) = |\Nf|+|\Nb|$.
Finally, we define the maximum degree $\Delta = \max_v\Delta(v)$.


A (simple) \emph{path} $P$ in $G$ is an acyclic sequence of nodes  $\crl*{u_1u_2u_3\ldots u_k}$ with $(u_i,u_{i+1})\in E$~\footnote{By acyclic, we refer to the fact that $u_i\neq u_j$ for any $i\neq j$ and $u_i,u_j\in P$.}. 
For any source-terminal pair $s,t\in V$, we denote by $\calP_{s,t}$ the set of all paths from $s$ to $t$.

The length $\ell(P)$ is the sum of edge lengths in $P$; by convention, a path with a single node $v$ has length zero.
For $s,t\in V$, the distance from $s$ to $t$, denoted $\dist(s,t)$, is the smallest length among all paths $P$ that start at $s$ and terminate at $t$.
The distance from a node $v$ to a path $P$, denoted $\dist(v,P)$, is measured as the minimum distance from $v$ to a node $w\in P$.
The distance to $v$, $\dist(P,v)$, is defined analogously~\footnote{Note that  $\dist(P,v)$ and  $\dist(v,P)$ need not be the same as the graph is directed.}.
We denote the shortest $(s,t)$-path (if it exists) as $P(s,t)$, and denote the set of all shortest paths in $G$ as $\calP$.



Analogous to length, we define the cost $c(P)$ as the sum of edge costs on path $P$. 
Our goal in this work is to develop a data structure that enables fast computation of \emph{efficient paths}
\begin{definition}
\textbf{(Efficient Paths)} A path $P\in \calP_{s,t}$ is said to be \emph{efficient} if there is no other path $P'\in \calP_{s,t}$ such that $\ell(P')\leq \ell(P)$ and $c(P')\leq c(P)$ with at least one inequality strict.
\end{definition}
We denote the set of all efficient paths as $\PE$, and define the \emph{Pareto frontier} from $s$ to $t$ as $\calP_{s,t}\cap\PE$.
Observe that every subpath of an efficient path is also efficient (if not, we could improve the path by replacing the subpath).


For $r>0$ and $v\in V$, we define the \emph{forward and reverse balls of radius $r$} by $\Bf_r(v)\defeq\{u\in V: \dist(v,u)\leq r\}$ and $\Bb_r(v)\defeq\{u\in V: \dist(u,v)\leq r\}$, and denote $B_r(v)\defeq\Bf_r(v)\cup\Bb_r(v)$.
The graph $G$ is said to have a \emph{doubling dimension} $\alpha$ if, for any node $v$ and any $r>0$, the ball $B_{2r}(v)$ can be covered by at most $\alpha$ balls $B_r(w)$ of radius $r$. 


