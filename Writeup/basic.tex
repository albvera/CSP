%!TEX root = main_vldb.tex

We consider a directed graph $G=(V,E)$ with length function $\ell:E\to \N$.
A \emph{path} $P$ is an acyclic sequence of nodes $\crl*{u_1u_2u_3\ldots u_k}$ with $(u_i,u_{i+1})\in E$~\footnote{By acyclic, we refer to the fact that $u_i\neq u_j$}. 
The length $\ell(P)$ is the sum of edge lengths in $P$; by convention, a path with a single node $v$ has length zero.
For $s,t\in V$, the distance from $s$ to $t$, denoted $\dist(s,t)$, is the smallest length among all paths $P$ that start at $s$ and terminate at $t$.
For $r>0$ and $v\in V$, we define the \emph{forward and reverse balls} by $\Bf_r(v):=\{u\in V: \dist(v,u)\leq r\}$ and $\Bb_r(v):=\{u\in V: \dist(u,v)\leq r\}$.
We also denote $B_r(v):=\Bf_r(v)\cup\Bb_r(v)$.
We denote the shortest $(s,t)$-path (if it exists) as $P(s,t)$, and denote the set of all shortest paths in $G$ as $\calP$.

Let $c:E\to \N\cup\{0\}$ be the cost function.
Analogously as the length of a path $P$, the cost $c(P)$ is the sum of edge costs.
For any source-terminal pair $s,t\in V$, denote by $\calP_{s,t}$ the set of all simple paths from $s$ to $t$.
A path $P\in \calP_{s,t}$ is called efficient if there is no other path $P'\in \calP_{s,t}$ such that $\ell(P')\leq \ell(P)$ and $c(P')\leq c(P)$ with at least one inequality strict.
The set of all efficient paths is denoted $\PE$.
In particular, all paths in $\calP_{s,t}\cap\PE$ form the Pareto frontier from $s$ to $t$.
Observe that every subpath of an efficient path is also efficient.
Indeed, if that is not the case, we could improve the path by replacing the subpath.

