In order to describe the main structure used to construct HL, we start with a discussion on efficient paths.
Given a source-terminal pair $s,t$ and a budget $b$, we want to solve the problem
\begin{alignat*}{3}
&\text{min}   &&\quad \ell(P) \\
&\text{s.t. } &&\quad  c(P) \leq b\\
& &&\quad P \in \calP_{s,t}.
\end{alignat*}
Call $\dist(s,t|b)$ the minimum of this problem, potentially $+\infty$ if there is no feasible solution.
Note that a solution to $\dist(s,t|b)$ is not necessarily efficient, i.e., there could be another path with the same length and lower cost.
Additionally, the budget constraint may not be tight and there could be multiple minimizers. 
As a consequence, a direct approach with EPHS would not work; there are potentially many more solutions than efficient paths.

For a fixed parameter $B\in \N$, we construct a structure that allows to obtain $\dist(s,d|b)$ for any $b\leq B$.
Additionally, we will always answer with an efficient path, so there could be a ``surplus of budget''.
On an intuitive level, we create a graph with the following properties:
\begin{enumerate}
\item All paths starting from a given node are feasible, i.e., they satisfy the budget constraint.
\item Efficient paths become shortest paths.
\item The use of ``unnecessary resource'' is penalized, hence inefficient paths are not shortest.
\end{enumerate}

We give now the formal description of the augmented graph.
Consider nodes as pairs of the form $\pp{v,b}$. 
This encodes the information of remaining resource $b\geq 0$ and location $v\in V$.
A node is connected to neighbours (according to $E$) as long as the remaining resource of that transition is non-negative.
Finally, we create $n$ sink nodes, denoted $\pp{v,-1}$.

\begin{definition}
Given a network $(G,\ell)$ and $B\in\N$, the augmented version $(G^B,\ell)$, where $G^B=(V^B,E^B)$, is defined by
\begin{align*}
V^B &:=\{\pp{v,b}: v\in V, b=-1,0,1,\ldots,B\},\\
E^B &:=\{\pp{v,b}\pp{v',b'} : vv'\in E, b'=b-c_{vv'}, b'\geq 0\}\cup E^{B}_{-},\\
E^{B}_{-} &:= \{\pp{v,b}\pp{v,-1}: v\in V, B\geq b\geq 0\}.
\end{align*}
The lengths to sink nodes are $\ell(\pp{v,b},\pp{v,-1})=\frac{1}{b+1}$.
The other lengths are preserved, i.e., $\ell(\pp{v,b},\pp{v',b'})=\ell(vv')$.
\end{definition}

\begin{remark}
The edge costs are allowed to be zero, therefore the augmented graph may contain cycles.
On the other hand, the dynamic programming approach requires a DAG \todo{insert reference}; the topological order is what ensures the efficiency of the algorithm.
\end{remark}

\begin{definition}
Let $P=\pp{v_1,b_1}\pp{v_2,b_2}\ldots\pp{v_k,b_k}$ be a path in $G^B$.
The projection of $P$, denoted $\bar P$, is the path in $G$ induced by $P$. 
Formally, in the case $b_k\geq 0$, $\bar P:=v_1v_2\ldots v_k$ and $\bar P:=v_1v_2\ldots v_{k-1}$ in the case $b_k=-1$. 
\end{definition}


\begin{proposition}\label{prop:shorteffic}
A shortest path from $\pp{s,b}$ to a sink node $\pp{t,-1}$ projects to an efficient path in $G$ solving $\dist(s,t|b)$. 
\end{proposition}
\begin{proof}
Let $P$ be the shortest path from $\pp{s,b}$ to $\pp{t,-1}$.
To reach $\pp{t,-1}$, $P$ must pass trough some $\pp{t,b'}$, $b'\geq 0$.
By construction, $P$ consumes $b-b'$ units of resource, hence it is feasible.
Note that $\bar P$ is the shortest among $(s,t)$-paths with cost $b-b'$.

Assume, by way of contradiction, that $\bar P$ is not efficient.
As $\bar P$ is the shortest using $b-b'$ units of resource, there exists $P'$ such that $\ell(\bar P')\leq \ell(\bar P)$ and $c(\bar P')< c(\bar P)$.
It must be that $P'$ passes through $\pp{t,b''}$, with $b''>b'$.
We argue that, in this case, $P$ would not be a shortest path to $\pp{t,-1}$.
Indeed, 
\[
\ell(P')=\ell(\bar P')+\frac{1}{1+b''}
\leq \ell(\bar P) +\frac{1}{1+b''}
< \ell(\bar P) +\frac{1}{1+b'},
\]
where the last expression is exactly $\ell(P)$.
\end{proof}

Using the previous result, we can construct SPHS in the augmented graph using the EPHS of the original graph.
It is interesting that, assuming doubling dimension $\alpha$ and $\beta$-witness, we obtain sparsity of $\alpha^{\ceil{\log_2\beta}} hB$ and thus query times of $O(\alpha^{\ceil{\log_2\beta}} hB\log D)$.
This is much better than the $O(Bn\Delta)$ of dynamic programming, but it is still pseudo-polynomial.
We observe that the query time can be made strongly polynomial if and only if the Pareto frontier contains at most a poly-logarithmic number of points.
See Section~\ref{sec:frontier} for a further discussion on this.
 
In $G^B$ we are not interested in every shortest path, but only on those ending in sink nodes, since these are projected to efficient paths. 
Let $\calP^B$ be such system, i.e., all shortest paths in $G^B$ ending in a sink node.
A hitting set for $\PE$ can be used to obtain a hitting set for $\calP^B$, but, since we require more information by augmenting the graph, the sparsity increases.
 
\begin{proposition}
Assume the system $\PE$ admits $(\hc,r)$-EPHS.
Then, $\calP^B$ admits $(\hc B,r)$-SPHS.
\end{proposition}
\begin{proof}
We define a set $C^B$ and prove that hits $\calP_r^B$ and that it is locally sparse.
Call $C$ the $(\hc,r)$-EPHS for $\PE$.
Define
\begin{equation}\label{eq:hitset}
C^B:=\{\pp{v,b}: v\in C, v \text{ hits }\bar P\in\calP_r^B, c(\bar P)=b\leq B \}.
\end{equation}

By Proposition~\ref{prop:shorteffic}, we know that shortest paths are efficient, hence $C^B$ hits all the desired paths.
Finally, we prove local sparsity.
Take any node $\pp{u,b}$ and observe that
\begin{align*}
\Bf_{2r}(\pp{u,b}) &= \{\pp{v,x}: \exists P\in\calP_{u,v}, \ell(P)\leq 2r, c(P)=b-x\} \\
&\subseteq \{\pp{v,x}: v\in \Bf_{2r}(u), x\leq b\} .
\end{align*}
We know that $\card{\Bf_{2r}(u)\cap C}\leq \hc$, therefore $\card{\Bf_{2r}(\pp{u,b})\cap C^B}\leq\hc b\leq \hc B$.
A similar argument shows the sparsity for the backward ball.
\end{proof}

\begin{remark}
The proof shows a stronger result.
In Equation~(\ref{eq:hitset}) we see that the sparsity around the node $\pp{u,b}$ is $\hc b$.
This will be key for the query time guarantee.
\end{remark}

The last result shows how SPHS translate to the augmented graph.
Since the notion of HD is much stronger, it is not always possible to make the analogous between two path systems.
Surprisingly, in this case we are able to relate the HD of both path systems.

\begin{proposition}\label{prop:HDaugmented}
If the HD of the system $\PE$ is $h_c$, then the HD of the system $\calP^B$ is $Bh_c$.
\end{proposition}
\begin{proof}
Fix $r>0$ and $\pp{v,b}\in V^B$ .
Let $H_{v,r}\subseteq V$ be the set hitting $S_r(v,\PE)$ and define $H:=H_{v,r}\times\{0,1,\ldots,B\}$.
We show that $H$ hits $\Sf_r(\pp{v,b},\calP^B)$.

Take $P\in\Sf_r(\pp{v,b},\calP^B)$.
Since $\dist(\pp{v,b},P)\leq 2r$, it holds $\dist(v,\bar P)\leq 2r$, therefore $\bar P\in \Sf_r(v,\PE)$.
Finally, $H_{v,r}$ hits $\bar P$, thus $H$ hits $P$.
A similar argument shows that $H$ hits $\Sb_r(\pp{v,b},\calP^B)$.
\end{proof}