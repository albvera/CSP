We use a variant of HL to find the efficient paths.
There is a subtle difference, we are only interested in paths ending in a sink node.
We explain how to create the HL and use them to answer the value of $\dist(s,t|b)$.
Afterwards we show how to use the previous to reconstruct the desired path in linear time.

Nodes $\pp{v,b}$ have associated a forward hub $\Lf(\pp{v,b})\subseteq V^B$.
A sink node has a backward hub $\Lb(u^-)\subseteq V^B$.
The hubs must satisfy the cover property: for every pair $\pp{s,b}$, $b\geq 0$, and $t^-$, there exists a node in $\pp{u,x}\in L(\pp{s,b})^+\cap L(t^-)^-$ such that $\pp{u,x}$ is in the shortest path from $\pp{s,b}$ to $t^-$.
As before, if the intersection contains many nodes, we take the one minimizing the quantity $\dist(\pp{s,b},\pp{u,x})+\dist(\pp{u,x},t^-)$.
Additionally, if there is no such $\pp{u,x}$, then there exists no path, i.e., $\dist(s,t|b)=\infty$.

The algorithm to construct the path is very simple if, as discussed in Section~\ref{sec:path_oracl}, we also store the first hop to every node in the hub.
Assume we can answer queries of the form $SP(\pp{s,b},t^-)$ and output either ``not possible'' or a node $\pp{v,x}$ in the SP and the budget consumed.
Consider the case when all the budget was used.
We can find a node in the SP from $\pp{s,b}$ to $\pp{v,x}$ by running a query $SP(\pp{s,b-x},v^-)$.
This process can be repeated until the path is recovered.
In the other hand, if there was a surplus of budget, we adjust $b$ to the smaller quantity; the adjustment is necessary only in the very first query, the subsequent queries will always consume all the budget.

\subsubsection{Query Time and Data Requirements}
We now prove that the desired HL exist and show how much extra information is needed.
It is interesting to note that the augmented graph is only a theoretical construction.
From a computational point of view, we can obtain the HL using only the EPHS of the original graph.
Nevertheless, the augmented graph allows us to prove that all the pieces fit together.

\begin{theorem}\label{theo:HLeff}
Assume the system $\PE$ admits an $(h_c,r)$-EPHS for every $r$.
Then, we can construct HL such that, for every $u\in V$ and $b\geq 0$, $\card{L(\pp{u,b})^+}\leq bh_c\log D$ and $\card{L(u^-)^-}\leq Bh_c\log D$. 
\end{theorem}
\begin{proof}
For $i=0,1,\ldots,\log D$, let $C_i$ be an $(h_c,2^{i-1})$-EPHS.
Now augment $C_i$ to $C_i^B$ as in Equation~(\ref{eq:hitset}).
Define the HL by
\begin{align*}
L(\pp{v,b})^+ &:= \bigcup_{i=1}^{\log D} C_i^B\cap \Bf_{2^i}(\pp{v,b})\\
L(u^-)^-  &:= \bigcup_{i=1}^{\log D} C_i^B\cap \Bb_{2^i}(u^-).
\end{align*}
The cover property is proved as in Theorem~\ref{theo:construct_hl}, we just bound the size here.
The number of elements in each forward hub is as claimed because $ \card{C_i^B\cap \Bf_{2^i}(\pp{v,b})}\leq bh_c$.
For a backward hub we use that
\begin{align*}
\Bb_{2^i}(u^-) &= \{\pp{v,x}: \exists P\in \calP_{v,u}, c(P)=x, \ell(P)\leq 2^i\}\\
&\subseteq \Bb_{2^i}(u)\times \{0,1,\ldots,B\}.
\end{align*}
Thus, $\Bb_{2^i}(u^-)\cap C_i^B\leq Bh_c$.
\end{proof}


\begin{proposition}
Using the HL given by Theorem~\ref{theo:HLeff}, we can implement queries for $s,t,b$ in time $O(b h_c\log D)$.
The total space requirement is $O(nB \cdot Bh_c\log D)$.
\end{proposition}
\begin{proof}
The backward hub for a sink $u^-$ has at most $Bh_c\log D$ nodes.
For each such node $\pp{v,b}\in\Lb(u^-)$ we store: the distance to the sink $\dist(v,u|b)$ and the efficient cost, i.e., we know how much surplus there is.
In the case of forward hub for a node $\pp{u,b}$ we store only distances.

To perform a query for $s,t,b$, we look only for nodes $\pp{u,x}$ with $x\leq b$ in the backward hub of $t^-$ \anote{this requires a data structure to know the level of each node, but I guess it's clear from the context}.
As we compare two ordered lists of size at most $h_cb\log D$, the result follows.
\end{proof}
