We now describe how to use the augmented graph $G^B$ to get hub-labels that allow us to find efficient paths for any query $SP(\pp{s,b},t^-)$.
This is similar to HL construction for shortest-paths (cf. Section \ref{ssec:hldef}); a subtle difference however is that we are only interested in paths ending in a sink node.
%Afterwards we show how to use the previous to reconstruct the desired path in linear time.

We associate each node $\pp{v,b}$ with a forward hub-label $\Lf(\pp{v,b})\subseteq V^B$, and \emph{only sink nodes} $u^-$ with a reverse hub $\Lb(u^-)\subseteq V^B$.
The required cover property is now as follows: for every pair $(\pp{s,b},t^-)$, there exists a node in $\pp{u,x}\in L(\pp{s,b})^+\cap L(t^-)^-$ such that $\pp{u,x}$ is in the shortest path from $\pp{s,b}$ to $t^-$.
If $L(\pp{s,b})^+\cap L(t^-)^-=\emptyset$, then is no feasible path, i.e., $\dist(s,t|b)=\infty$; if the intersection contains multiple nodes, we take the one minimizing $\dist(\pp{s,b},\pp{u,x})+\dist(\pp{u,x},t^-)$.
Finally, we can augment the hub-labels with the next-hop node, and compute the entire path recursively.

%The algorithm to construct the path is very simple if, as discussed in \todo{reference}, we also store the first hop to every node in the hub. Assume we can answer queries of the form $SP(\pp{s,b},t^-)$ and output either ``not possible'' or a node $\pp{v,x}$ in the SP and the budget consumed. Consider the case when all the budget was used. We can find a node in the SP from $\pp{s,b}$ to $\pp{v,x}$ by running a query $SP(\pp{s,b-x},v^-)$. This process can be repeated until the path is recovered. In the other hand, if there was a surplus of budget, we adjust $b$ to the smaller quantity; the adjustment is necessary only in the very first query, the subsequent queries will always consume all the budget.

\subsubsection{Query Time and Data Requirements}
Putting things together, we now show how we can construct hub-labels for answering CSP queries, such that their preprocessing time and storage is parameterized by the constrained HD $h_c$.
Note that we use the augmented graph primarily as a theoretical construction; from a computational point of view, we can obtain the HL using only an EPHS of the original graph.
%Nevertheless, the augmented graph allows us to prove that all the pieces fit together.

\begin{theorem}
\label{theo:HLeff}
For graph $G(V,E,\ell,c)$, given a multi-scale EPHS $\crl*{C_i:i=0,1,\ldots,\log D}$ (where for each $i$, $C_i$ is an $(h_c,2^{i-1})$-EPHS), we can construct hub labels for answering CSP queries such that, for every $u\in V$ and $b\geq 0$, $\card{L(\pp{u,b})^+}\leq bh_c\log D$ and $\card{L(u^-)^-}\leq Bh_c\log D$. 
\end{theorem}
\begin{proof}
As in Equation~\eqref{eq:hitset}, we define the HL as
\begin{align*}
L(\pp{v,b})^+ &\defeq \bigcup_{i=1}^{\log D} C_i^B\cap \Bf_{2^i}(\pp{v,b})\\
L(u^-)^-  &\defeq \bigcup_{i=1}^{\log D} C_i^B\cap \Bb_{2^i}(u^-).
\end{align*}
The cover property is proved in an identical manner as in Proposition~\ref{theo:construct_hl}; we now bound the hub size.
For forward hubs, the size follows from observing that $ \card{C_i^B\cap \Bf_{2^i}(\pp{v,b})}\leq bh_c$.
For a reverse hub we use that
\begin{align*}
\Bb_{2^i}(t^-) &= \{\pp{s,x}: \exists P\in \Pst, c(P)=x, \ell(P)\leq 2^i\}\\
&\subseteq \Bb_{2^i}(t)\times \{0,1,\ldots,B\}.
\end{align*}
Thus, $\Bb_{2^i}(t^-)\cap C_i^B\leq Bh_c$.
\end{proof}

This immediately gives us the following corollary for the storage and query time.
\begin{corollary}
Using the HL given by Theorem~\ref{theo:HLeff}, we can implement queries for $s,t,b$ in time $O(b h_c\log D)$.
The total space requirement is $O(nB \cdot Bh_c\log D)$.
\end{corollary}

%\begin{proof}
%The reverse hub for a sink $u^-$ has at most $Bh_c\log D$ nodes. For each such node $\pp{v,b}\in\Lb(u^-)$ we store: the distance to the sink $\dist(v,u|b)$ and the efficient cost, i.e., we know how much surplus there is. In the case of forward hub for a node $\pp{u,b}$ we store only distances.

%To perform a query for $s,t,b$, we look only for nodes $\pp{u,x}$ with $x\leq b$ in the reverse hub of $t^-$ \anote{this requires a data structure to know the level of each node, but I guess it's clear from the context}. As we compare two ordered lists of size at most $h_cb\log D$, the result follows.
%\end{proof}
