\begin{proofof}{Proposition~\ref{prop:poly_lshs}}
Denote $S_r(v):=\Sf_r(v,\calQ)\cup\Sb_r(v,\calQ)$. 
Observe that, for fixed $v\in V$, the set system $(E,\{\pi(Q):Q \in S_r(v)\})$ admits a hitting set of size $h\Delta$.
Indeed, we know that exists $H_{v,r}\subseteq V$, $\card{H_{v,r}}\leq h$, hitting every path in $\Sf_r(v,\calQ)$ and in $\Sb_r(v,\calQ)$.
The desired hitting set consists of all the edges adjacent to a node in $H_{v,r}$.

If the minimum size of a set system is $s$ and the VC-dimension is $d$, then the algorithm in \cite{vc_dim_hitting} obtains, in polynomial time, a hitting set of size at most $\Or(sd\log(sd))$.
In particular, we can use the algorithm to obtain a set $\tilde F_{v,r}\subseteq E$, of size at most $h'=\Or(h\Delta\log(h\Delta))$, hitting the set system $(E,\{\pi(Q):Q \in S_r(v)\})$ .

Consider the set $F_{v,r}\subseteq V$ that contains all the endpoints of edges in $\tilde F_{v,r}$.
It follows that $F_{v,r}\subseteq V$ can be obtained in polynomial time and is a hitting set for $S_r(v)$ of size $\card{F_{v,r}}\leq 2h'$.

Assume for now that we know the value of $h$.
Note that the value $h'$ can be computed from $h$ and the guarantee given by the oracle, i.e., the constant inside the big-O.
We construct the $(2h',r)$-LSHS iteratively.
At each iteration $i$ we maintain the following invariant: $C_i$ hits every path in $\calQ_r$.
In an iteration we check if $C_i$ is locally sparse, if not, we strictly reduce the cardinality of $C_i$ while maintaining the invariant.
Start with $C_0=V$. 
Let $B_{2r}(v):=\Bf_{2r}(v)\cup \Bb_{2r}(v)$.
Assume $v\in V$ is such that $\card{B_{2r}(v)\cap C_i}>2h'$ and let $C_{i+1}:=(C_i\setminus B_{2r}(v))\cup F_{v,r} $.
The cardinality strictly decreases and we only need to check the invariant.
Consider the paths hit by nodes removed in $C_i$, this set is
\begin{align*}
&\hspace{-1cm}\{Q\in\calQ_r: Q\cap C_i\cap B_{2r}(v)\neq \varnothing\}\\
&\subseteq \{Q\in\calQ_r: Q\cap B_{2r}(v)\neq \varnothing\} \subseteq S_r(v).
\end{align*}
Since $F_{v,r}$ hits $S_r(v)$, the proof is completed.

If we do not know the value of $h$, we can do a doubling search for $h'$. 
Indeed, if the guess of $h'$ is low, then at some point it could be that $\card{F_{v,r}}>2h'$, then we double $h'$ and restart the process.
\end{proofof}
