The construction is similar to HL for shortest-paths (cf. \cref{ssec:hldef}).
A subtle difference is that we are only interested in paths ending in a sink node.
Each node $\pp{v,b}$ has a forward hub label $\Lf(\pp{v,b})\subseteq V^B$, and \emph{only sink nodes} $u^-$ have a reverse hub $\Lb(u^-)\subseteq V^B$.
The cover property must be satisfied for every $\pp{s,b}$ and $t^-$.
Finally, if we want to reconstruct the path, we can proceed similarly as in ~\cref{ssec:hldef}; we can augment the hub labels with the next-hop node, and compute the entire path recursively.
Putting things together, we now show how we can construct hub labels for answering CSP queries, such that their preprocessing time and storage is parameterized by the CHD $h_c$.

\subsubsection{Query Time and Data Requirements}

\begin{theorem}
\label{theo:HLeff}
For a network $(G,\ell,c)$, given a multi-scale EPHS $\crl*{C_i:i=0,1,\ldots,\log D}$, where $C_i$ is an $(h_c,2^{i-1})$-EPHS, we can construct hub labels for answering CSP queries such that, for every $u\in V$ and $b\geq 0$, $\card{L(\pp{u,b})^+}\leq (b+1)h_c(1+\log D)$ and $\card{L(u^-)^-}\leq (B+1)h_c(1+\log D)$. 
\end{theorem}
\begin{proof}
Create $C_i^B$ as in \cref{eq:hitset}.
Define $L(\pp{v,b})^+ \defeq \bigcup_{i=1}^{\log D} C_i^B\cap \Bf_{2^i}(\pp{v,b})$ and $L(u^-)^-  \defeq \bigcup_{i=1}^{\log D} C_i^B\cap \Bb_{2^i}(u^-)$.
The cover property is proved similarly as in \cref{theo:construct_hl}; we are left to bound the hub size.
For a reverse hub we use that $\Bb_{2^i}(t^-) = \{\pp{s,x}: \exists P\in \Pst, c(P)=x, \ell(P)\leq 2^i\}
\subseteq \Bb_{2^i}(t)\times \{0,1,\ldots,B\}$.
Thus, $\Bb_{2^i}(t^-)\cap C_i^B\leq (B+1)h_c$.
For forward hubs, the size follows from observing that $ \card{C_i^B\cap \Bf_{2^i}(\pp{v,b})}\leq (b+1)h_c$.
\end{proof}

This immediately gives us the following corollary for the storage and query time.
\begin{corollary}
Using the HL given by \cref{theo:HLeff}, we can implement queries for $s,t,b$ in time $O((b+1) h_c\log D)$.
The total space requirement is $\Or(nB \cdot Bh_c\log D)$.
\end{corollary}