We show how the preprocessing requirements for HL construction can be parameterized in terms of $h_c$. For this, we need to modify similar arguments for shortest paths in \cite{highway2013} to handle efficient paths.
We recall the concept of VC-dimension and latter explain how this notion allows to obtain hitting sets.
A set system $(X,\calX)$ is a ground set $X$ together with a family $\calX\subseteq 2^X$.
A set $Y\in \calX$ is shattered if $\{Z\cap Y:Z\in\calX\}=2^Y$.
If $d$ is the smallest integer such that no $Y\in\calX$ with $\card{Y}=d+1$ can be shattered, then this number $d$ is the VC-dimension of $(X,\calX)$.

The critical observation in \cite{highway2013} is that the set system of \emph{unique} shortest paths has a VC-dimension of $2$.~\footnote{
Formally, paths $v_1v_2\ldots v_k$ are mapped to sets $\{v_1,\ldots,v_k\}$. If $G$ is undirected, each subset represents exactly two paths, the $(s,t)$-path and $(t,s)$-path. Therefore, a hitting set for this set system corresponds to a hitting set for $\PS$.}
To see this, take the shortest path $abc$, represented by $\{a,b,c\}$. Because the paths are unique, $ac$ will not be a shortest path, hence $\{a,c\}$ can not be in the system and thus $\{a,b,c\}$ cannot be shattered.
This does not hold in directed graph, as both paths $abc$ and $ca$ may be shortest paths. However, we can overcome this by considering the ground set as $E$ instead of $V$, and mapping a path $e_1e_2\ldots e_k$ to $\{e_1,e_2,\ldots,e_k\}$.
Note that each $\{e_1,e_2,\ldots,e_k\}$ corresponds uniquely to one path (since we consider acyclic paths).
Let $\pi(Q)$ denote the set of edges in a path $Q$.
\begin{proposition}
Given a path system $\calQ$, the corresponding set system $(E,\{\pi(Q):Q\in\calQ\})$ has VC-dimension 2.
\end{proposition}
Note that this argument also can be used for shortest paths in undirected graphs to remove the uniqueness requirement.
Finally, polynomial-time preprocessing now follows from extending a similar argument as Theorem 8.2 in \cite{highway2013}.
We defer the proof to Appendix~\ref{sec:proofs}.

\begin{proposition}\label{prop:poly_lshs}
If $\calQ$ has HD $h$, then in polynomial time we can obtain a $(h',r)$-LSHS, where $h'=\Or(h\Delta\log(h\Delta))$.	
\end{proposition}