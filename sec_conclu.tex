We introduced a new network primitive, the Constrained Highway Dimension, and used it to parametrize the storage and running time of data structures that support fast CSP queries.
Our aim was to study when efficient SP computation yields a similar performance for the harder CSP problem.
We derived conditions under which this holds and we can compare the HD and CHD.
For the worst-case setting, the conditions are given by the partial-witness and for average-case by the milder bounded growth. 
Both conditions have intuitive interpretations in terms of the physical structure of the network. 

On the practical side, we validated our findings by developing algorithms that performed four orders of magnitude better on real-world networks, compared to standard techniques.
Our work is a first step in bridging the gap between SP and CSP algorithms and we believe our findings are promising for real-world applications.