We have shown that, for networks where we can compute shortest paths efficiently, we can also solve for constrained shortest paths in comparable time.
For this end, we introduced a network primitive, the Constrained Highway Dimension, and parametrize the storage and running time of our algorithms with it.
Furthermore, we proved that under practical assumptions this primitive is closely related to the Highway Dimension, which is known to parametrize shortest-path algorithms.
We also showed that under even weaker assumptions we can derive algorithms with the same guarantees in average.

On the practical side, we validated our findings by studying real-world networks.
Algorithms based on our ideas perform four orders of magnitude better than standard techniques.
Since we did not implement other state-of-the-art procedures, we believe that our results are promising for real-world applications.
