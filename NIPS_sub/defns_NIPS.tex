%!TEX root = main_nips.tex

\subsection{Basic Setting}
\label{ssec:basic}

We consider a directed graph $G=(V,E)$ with \emph{length function} $\ell:E\to \N_+$, and \emph{cost function} $c:E\to \N_+\cup\{0\}$.
For each node $v$, we denote its degree $\Delta(v)$ as the sum of the in-degree and out-degree.
Finally, we define the \emph{maximum degree} $\Delta = \max_v\Delta(v)$.


A simple \emph{path} $P$ in $G$ is an acyclic sequence of nodes  $u_1u_2u_3\ldots u_k$ with $u_iu_{i+1}\in E$.
For any source-terminal pair $s,t\in V$, we denote by $\Pst$ the set of all simple paths from $s$ to $t$.
Throughout this work, we only consider simple paths, which for brevity, we henceforth refer to as paths.

The length $\ell(P)$ is the sum of edge lengths in $P$. Note also that any path $P$ with more than one node has length at least $1$ (since $\ell:E\to \N_+$).
For $s,t\in V$, the distance from $s$ to $t$, denoted $\dist(s,t)$, is the smallest length among all paths $P\in\Pst$.
The distance from a node $v$ to a path $P$, denoted $\dist(v,P)$, is measured as the minimum distance from $v$ to a node $w\in P$.
The distance to $v$, $\dist(P,v)$, is defined analogously.\footnote{Note that  $\dist(P,v)$ and  $\dist(v,P)$ need not be the same as the graph is directed.}
We denote the shortest $(s,t)$-path (if it exists) as $P(s,t)$, and denote the set of all shortest paths in $G$ as $\PS$.
Finally, we define $D=\max_{P\in\PS}\ell(P)$ to be the diameter of $G$.


Analogous to length, we define the cost $c(P)$ as the sum of edge costs on path $P$. 
Our goal is to develop a data-structure that can solve the following \emph{Constrained Shortest-Path} (CSP) problem: Given a source-terminal pair $s,t$ and a budget $b$, we want to return a path $P$ that solves
\begin{align*}
\min_{P \in \Pst}  \quad \ell(P) \quad
\text{s.t. } \quad  c(P) \leq b 
\end{align*}
We define $\dist(s,t|b)$ to be the minimum of this problem.
If there is no feasible solution, we define $\dist(s,t|b)=\infty$.
Note that the CSP problem may have multiple solutions (with different feasible costs), as there could be several paths with the same length and cost lower than $b$.
To limit these solutions to those with minimal cost, we require that the path also be \emph{efficient}. 
\begin{definition}
\textbf{(Efficient Path)} A path $P\in \Pst$ is called \emph{efficient} if there is no other path $P'\in \Pst$ such that $\ell(P')\leq \ell(P)$ and $c(P')\leq c(P)$ with at least one inequality strict.
\end{definition}
We denote the set of all efficient paths as $\PE$, and define the \emph{Pareto frontier} from $s$ to $t$ as $\Pst\cap\PE$.
Observe that every subpath of an efficient path is also efficient (if not, we could improve the path by replacing the subpath).

For $r>0$ and $v\in V$, we define the \emph{forward and reverse balls of radius $r$} by $\Bf_r(v)\defeq\{u\in V: \dist(v,u)\leq r\}$ and $\Bb_r(v)\defeq\{u\in V: \dist(u,v)\leq r\}$, and also define $B_r(v)\defeq\Bf_r(v)\cup\Bb_r(v)$.
Graph $G$ is said to have a \emph{doubling dimension} $\alpha$ if, for any node $v$ and any $r>0$, the ball $B_{2r}(v)$ can be covered by at most $\alpha$ balls $B_r(w)$ of radius $r$. 

\subsection{Hitting sets and the highway dimension}
\label{ssec:hddef}

%!TEX root = main_vldb.tex

We now define the notions of hitting sets for path systems, and the highway dimension (HD), which we use to parametrize the performance of our algorithms. The highway dimension was introduced by Abraham et al.~\cite{highway2013,highway2010} for undirected graphs and the shortest-path set system. 
Our presentation closely follows these works.
However, we need to extend the definitions to directed graphs and general path systems for our purposes. 

We define a \emph{path system} $\calQ$ as any collection of paths.
Given a set $C\subseteq V$ and a path $Q$, we say that $C$ \emph{hits} $Q$ if some node in $Q$ belongs to $C$. Moreover, we say that $C$ is a \emph{hitting set for a path system} $\calQ$ if it hits every $Q\in\calQ$. Also, for any $r>0$, we say a path $Q$ is $r$-significant if $\ell(Q)>r$. 
For a given path system $\calQ$, we denote $\calQ_r$ as the set of all $r$-significant paths in $\calQ$.

Hitting sets are useful for compressing path systems. In particular, even if the hitting set is large, the extent to which a path system can be compressed depends on the \emph{local sparsity} of hitting sets with respect to \emph{significant paths} of $\calQ$.

\begin{definition}
	\label{def:lshs}
	\textbf{(Locally-Sparse Hitting Sets)} Given a path system $\calQ$ and $r>0$, an $(h,r)$ locally-sparse hitting set (or $(h,r)$-LSHS) is a set $C\subseteq V$ with two properties: 
	\begin{enumerate}[nosep]
		\item Hitting: $C$ is a hitting set for $\calQ_r$.
		\item Local sparsity: for every $v\in V$, $\abs*{B_{2r}(v)\cap C}\leq h$.
	\end{enumerate}
\end{definition}


The existence of $(h,r)$-LSHS enables the compression of path system $\calQ$ via the construction of \emph{hub labels} (cf. Section~\ref{ssec:hldef}). However, this existence does not guarantee the ability to efficiently compute it. 
To address this, Abraham et al.~\cite{highway2013} introduced the notion of the \emph{highway dimension} -- a property of the graph which ensures both existence and efficient computation of LSHS. 

To define the highway dimension, we first need two additional definitions:
for $v\in V, r>0$, we define the \emph{forward and reverse path-neighborhoods} with respect to path system $\calQ$ as: 
\[
S_r^+(v,\calQ) \defeq\crl*{Q\in\calQ_r: \dist(v,Q)\leq 2r} \quad\text{ and }\quad
S_r^-(v,\calQ) \defeq\crl*{Q\in\calQ_r: \dist(Q,v)\leq 2r}.
\]
As before, we have $S_r(v,\calQ) = S_r^+(v,\calQ)\cup S_r^-(v,\calQ)$. Now we can define the highway dimension (HD) of a path system $\calQ$. Essentially, the HD re-orders the sequence of qualifiers in the definition of $(h,r)$-LSHS: it requires the existence of a small hitting set for each individual neighborhood, rather than a single hitting set which is locally sparse. 
\begin{definition}
	\label{def:hddef}
	\textbf{(Highway Dimension)}
	A path system $\calQ$ has HD $h$ if, for every $r>0$, and for every $v\in V$, there exists a set $H_{v,r}\subseteq V$ such that $\card{H_{v,r}}\leq h$ and $H_{v,r}$ is a hitting set for $S_r^+(v,\calQ)\cup S_r^-(v,\calQ)$.
\end{definition}
The HD was defined in \cite{highway2013} for undirected graphs and the shortest-path system $\PS$; the above definition is the natural analog for directed graphs and general path systems.
As shorthand, we refer to the HD of $(G,\ell)$ as that of $\PS$. 
%Otherwise, it will be clear from the context the path system we are using.
To see that $HD\leq h$ is a more stringent requirement than the existence of an $(h,r)$-LSHS $C$, note that $C\cap B_{2r}(v)$ need not hit all the paths in $S_r(v,\calQ)$. 
However, if $G$ has $HD\leq h$, then this guarantees the existence of a $(h,r)$-LSHS.
\begin{proposition}
	If the path system $\calQ$ has HD $h$, then, $\forall\,r>0$, there exists an $(h,r)$-LSHS.
\end{proposition}
This follows directly from Theorem 4.2 in \cite{highway2013}, which showed it for the shortest-path system. More importantly, as we discuss in Section~\ref{sec:preproc}, if $G$ has HD$\leq h$, then this permits efficient computation of locally-sparse hitting sets.
In particular, for any $r$, a simple greedy algorithm gives an $(\Or(h\log n),r)$-LSHS, and a more complex algorithm (which is still poly-time) gives an $(\Or(h\Delta\log h\Delta),r)$-LSHS.


Finally, we note that for pedagogical reasons, Definition~\ref{def:hddef} differs slightly from \cite{highway2013}.
We consider a less restrictive definition of path neighborhoods (albeit for general path systems) that is appropriate for our needs. Consequently the highway dimension as returned by our definition is smaller than that of \cite{highway2013}.\footnote{In particular, unlike \cite{highway2013}, the HD of $G$ as per our definition is not an upper bound to the maximum degree $\Delta$ or the doubling dimension $\alpha$.}
We discuss the relation between the two definitions in detail in Appendix \ref{app:generalhd}. 


\subsection{Shortest-Paths via Hub Labels}
\label{ssec:hldef}

Two of the most successful data-structures enabling fast shortest path queries at scale are \emph{contraction hierarchies} (CH)~\cite{geisberger_ch_definition} and \emph{hub labels} (HL)~\cite{cohen_definition_hl}.
These are general techniques which always guarantee correct SP computation, but have no uniform storage/query-time bounds for all graphs.
However, Abraham et al.~\cite{highway2013} show that for a graph with HD $\leq h$, the preprocessing time, storage requirements and query time of particular implementations of the CH and HL algorithms can be bounded solely in terms of $h$, the maximum degree $\Delta$ and the diameter $D$. 
We now explain the construction for HL for directed graphs; for the construction of CH refer to our technical report~\cite{TechReport}.
HL was also shown to admit the best query-time guarantees in~\cite{highway2013}, consequently, we adapt the HL technique for scaling CSP computations in Section \ref{sec:chd}.

The basic HL technique is as follows:
Every node $v$ is associated with \emph{hub labels}, $L(v) = \crl*{\Lf(v),\Lb(v)}$, comprising of a set of forward hubs $\Lf(v)$ and reverse hubs $\Lb(v)$.
We also store $\dist(v,w),\dist(u,v)$ for every $w\in\Lf(v),u\in\Lb(v)$.
The hub labels are said to satisfy the \emph{cover property} if, for any $s\neq t\in V$, $\Lf(s)\cap\Lb(t)$ contains at least one node in $P(s,t)$.
In the case that $t$ is not reachable from $s$, it must be that $\Lf(s)\cap\Lb(t)=\varnothing$.

With the aid of the cover property, we can obtain $\dist(s,t)$ by searching for the minimum value of $\dist(s,w)+\dist(w,t)$ over all nodes $w\in\Lf(s)\cap\Lb(t)$.
If the hubs are sorted by ID, this can be done in time $\Or(\card{\Lf(s)}+\card{\Lb(t)})$ via a single sweep.
Moreover, by storing the second node in $P(s,w)$ for each $w\in \Lf(s)$, and the second-last node in $P(w,t)$ for each $w\in \Lb(t)$, we can also recover the shortest path.
Now, each time we run a HL query, at least one new node $w\in P(s,t)$, $w\neq s,t$, is returned; we can then recurse to find subpaths $P(s,w)$ and $P(w,t)$.
Note that we need to store this extra information, as otherwise, we could have $\Lf(s)\cap\Lb(t)=\{s\}$.
Let $\Lm \defeq \max_v\abs*{L(v)}$ be the size of the maximum HL.
The per-node storage requirement is $\Or(\Lm)$, while the query time is $\Or(\Lm\ell(P(s,t)))$.

To construct hub labels with guarantees on preprocessing time and $\Lm$, we need the additional notion of a \emph{multi-scale LSHS}. 
We assume that graph $(G,\ell)$ admits a collection of sets $\crl*{C_i: i=1,\ldots,\log D}$, such that each $C_i$ is an $(h,2^{i-1})$-LSHS.
Given such a collection, we can now obtain small HL.
We outline this construction for directed graphs, closely following the construction in \cite{highway2013} (Theorem 5.1) for the undirected case.
\begin{proposition}
	\label{theo:construct_hl}
	For $(G,\ell)$, given a multi-scale LSHS collection $\crl*{C_i:i=1,\ldots,\log D}$, where each $C_i$ is an $(h,2^{i-1})$-LSHS, we can construct hub labels of size at most $h\log D$.
\end{proposition}
\begin{proof}
	For each node $v$, we define the hub label $L(v)$ as
	\[
	\Lf(v)\defeq  \bigcup_{i=1}^{\log D}C_i\cap \Bf_{2^i}(v) \quad\text{ and }\quad
	\Lb(v)\defeq \bigcup_{i=1}^{\log D}C_i\cap \Bb_{2^i}(v).
	\]
	Since each $C_i$ is an $(h,2^{i-1})$-LSHS which we intersect with balls of radius $2\cdot 2^{i-1}$, every set in the union contributes at most $h$ elements and the maximum size is as claimed.
	
	To prove the cover property, we note that, if $t$ is not reachable from $s$, by definition $\Lf(s)\cap\Lb(t)=\varnothing$.
	This is because the elements in $\Lf(s)$ are reachable from $s$ and the elements in $\Lb(t)$ reach $t$.
	On the other hand, when $P(s,t)$ exists, let $i$ be such that $2^{i-1}<\ell(P(s,t))\leq 2^i$.
	Now any point in the path belongs to both $\Bf_{2^i}(s)$ and $\Bb_{2^i}(t)$, and hence $C_i\cap P(s,t)$  (which is not empty since $C_i$ hits all SP of length $\geq 2^{i-1}$) is then in both hubs, which shows the result.
\end{proof}


Finally, we need to compute the desired multi-scale LSHS in polynomial time.
As we mentioned before, a greedy algorithm returns a $\Or(h\log n)$ approximation to any $(h,r)$-LSHS, however, if the HD $\leq h$, this can be improved to obtain a $\Or(h\Delta\log(h\Delta))$ approximation.
This argument, which extends Corollary 7.3 in \cite{highway2013}, is presented in Section~\ref{sec:preproc}.
A more subtle point is that the resulting algorithm, even though admitting a polynomial time guarantee, is impractical for large networks.
In Section~\ref{sec:numeric}, we discuss heuristics that work better in practice.