Motivated by the needs of modern on-demand transportation platforms for robust travel-time estimates, we study the problem of computing constrained shortest paths (CSP) at scale via preprocessing and network augmentation techniques.
%Preprocessing techniques and network augmentation have led to dramatic improvements in the speed and scalability of shortest path (SP) computations.
Our work makes two contributions in this regard:\\
1. We propose a preprocessing and network augmentation algorithm for scalable CSP queries, and show how the performance of our algorithm can be parametrized in terms of a new network primitive, the \emph{constrained highway dimension}. 
This development is analogous to recent work which established the highway dimension as the appropriate primitive for characterizing the performance of existing shortest-path (SP) algorithms. 
Our main theoretical contribution is in deriving conditions relating the two notions, thereby providing a characterization of networks where CSP and SP queries are of comparable hardness.\\
%We propose a preprocessing and network augmentation technique for scalable CSP queries, which builds on the \emph{hub labels} approach for shortest path (SP) computation. The performance of our algorithms can be parametrized in terms of a new network primitive, the \emph{constrained highway dimension}, analogous to the highway dimension for SP queries; we show how the two notions can be related, thereby providing theoretical characterization of networks which enable fast CSP queries.\\
2. We develop practical algorithms for scalable CSP computation, augmenting our theory with additional network clustering and ranking heuristics. We run these algorithms on real-world datasets to validate our theoretical findings. Our techniques are orders of magnitude faster in experiments compared to existing approaches, while requiring only limited additional storage and preprocessing.