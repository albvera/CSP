Motivated by the needs of modern on-demand transportation platforms for robust travel-time estimates, we study the problem of computing constrained shortest paths (CSP) at scale via preprocessing and network augmentation techniques.
%Preprocessing techniques and network augmentation have led to dramatic improvements in the speed and scalability of shortest path (SP) computations.
Our work makes two contributions in this regard:\\
1. We propose a preprocessing and network augmentation technique for scalable CSP queries, which builds on the \emph{hub labels} approach for shortest path (SP) computation.
The performance of our algorithms can be parametrized in terms of a new network primitive, the \emph{constrained highway dimension}, analogous to the highway dimension for SP queries;
%We adapt existing preprocessing techniques for fast SP queries (in particular, hub labels) to handle CSP queries; we also extend the \emph{highway-dimension}, a measure of graph structure that enables SP speedup techniques, to an analogous \emph{constrained highway dimension} for CSP queries.
we show how the two notions can be related, thereby providing theoretical characterization of networks which enable fast CSP queries.\\
2. We develop practical algorithms for scalable CSP computation, augmenting our theory with additional network clustering and ranking heuristics. We evaluate our algorithm on real-world datasets and show that our algorithms are orders of magnitude faster compared to existing techniques, and also have small per-node storage and good preprocessing times, even on a single machine.